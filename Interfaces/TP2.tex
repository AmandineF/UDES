\documentclass[12pt]{article}
\usepackage[T1]{fontenc}
\usepackage[utf8]{inputenc}
\usepackage[french]{babel}
\usepackage[right=2.5cm, bottom=2.5cm,top=2.5cm, left=2.5cm]{geometry}
\title{\vspace{\fill} Interface et Multimédia \\ ~\textbf{IFT-215} \\~\\ Travail Pratique 2}
\author{Amandine Fouillet - 14 130 638 ~\\ Frank Chassing - 14 000 00 0~\\ Laurent Machin - 14 000 000}
\date{\today \vspace{\fill}}
\begin{document}
\maketitle
\newpage

\section{Trouver un titre}

\section{Analyse des tâches}
Maintenant que les deux utilisateurs ainsi que leur environnement ont été présentés, nous allons dans cette partie détailler les tâches qu'ils effectuent pour rester en contact. Afin de structurer cette analyse, nous allons diviser ces tâches en deux catégories : les actions réalisées sur un ordinateur fixe ou portable et les actions réalisées sur un smartphone. Au sein de ces deux catégories, nous détaillerons à la fois les tâches effectuées par l'étudiant et celles effectuées par ses parents qui sont pratiquement similaires.

\subsection{Sur un ordinateur}
Pour commencer, nous allons détailler les tâches réalisées sur un ordinateur. Pour les parents ces actions seront donc plutôt réalisées dans leur maison, un endroit calme et familier tandis que pour l'étudiant, ces tâches peuvent être réalisées dans sa chambre étudiante mais aussi en cours ou à la bibliothèque. Nous allons étudier quatre catégories de communication qu'utilisent l'enfant et ses parents pour communiquer à travers un ordinateur connecté à internet. Nous détaillerons tout d'abord les méthodes de communication directe comme les chats et les appels vidéo et audio. Puis, nous analyserons les tâches de communication indirecte comme le partage de photos et les blogs.~\\

Actuellement, on peut dire que la tâche la plus fréquemment effectuée pour communiquer au sein d'une famille est l'écriture de messages textuels à travers des chats. Pour réaliser cette tâche sur un ordinateur, l'enfant et ses parents peuvent utiliser plusieurs interfaces de messageries instantanées comme le logiciel Skype ou encore Facebook Messenger. Sans oublier l'outil que tout le monde utilise aujourd'hui pour échanger des messages : le mail. Ainsi, l'étudiant se connecte sur son compte Skype, Facebook ou sa boîte mail puis, selon l'interface, il sélectionne ses parents ou tape leur adresse. Une fois cette étape réalisée, il peut taper son message au clavier et l'envoyer. Presque instantanément, s'ils se trouvent sur l'ordinateur, ses parents sont averti qu'un message est arrivé par une notification textuelle ou sonore. Ils peuvent alors répondre et renvoyer un message à leur tour, et ainsi de suite. Bien entendu, cette tâche peut être réalisée dans l'ordre inverse : les parents peuvent écrire à leur enfant en premier. Cette action se termine quand un des deux participants (l'enfant ou les parents) doit quitter l'ordinateur, ils se préviennent mutuellement et ferment simplement la fenêtre du chat qu'ils ont utilisé. La fréquence à laquelle ils effectuent cette tâche dépend des personnes, des jours et de l'emploi du temps de chacun mais c'est la tâche qui est la plus effectuée par nos utilisateurs pour garder le contact. En effet, c'est un moyen simple et rapide pour contacter l'autre personne et donner des nouvelles. De plus, ce système peut fonctionner unilatéralement. C'est-à-dire que si l'un des participants n'est pas à son ordinateur, l'autre peut tout de même laisser un message pour donner de ses nouvelles par exemple et celui qui était absent reçoit le message lorsqu'il rallume son ordinateur et peut y répondre même des heures après.~\\

Les outils de chats instantanés sont des merveilleux outils pour garder le contact mais il est parfois nécessaire, pour garder des liens, de se téléphoner ou de se voir de temps à temps. Pour cela, internet regorge de possibilités mais nos utilisateurs, comme beaucoup de familles séparées, utilisent surtout le logiciel Skype pour se contacter. 
Pour ce faire, ils doivent tout d'abord se mettre d'accord sur la date et l'heure à laquelle ils veulent se contacter car les deux utilisateurs doivent être devant leur ordinateur pour que la communication s'établisse. Ainsi, après avoir décidé d'une sorte de rendez-vous, chacun ouvre le logiciel sur son ordinateur et se connecte sur son compte Skype. Un des deux utilisateurs appelle l'autre qui répond positivement à l'appel de son côté. Avec Skype, il y a trois possibilités :  l'appel audio simple comme une liaison téléphonique, l'appel vidéo sans son (on communique alors par chat) et enfin, le plus fréquemment utilisé l'appel vidéo avec son. Les utilisateurs choisissent la méthode de communication en fonction de l'environnement dans lequel ils se trouvent. En effet, s'ils ne sont pas dans un endroit clos il est parfois délicat de mettre le son, ou alors il est mieux de ne pas mettre la vidéo si la connexion internet laisse à désirer. Une fois que le contact est établi, la conversation se déroule comme un coup de téléphone normal et la tâche prend fin une fois que la conversation est terminée. Cette méthode de communication est beaucoup utilisée pour garder le contact dans une famille parce qu'il est parfois plus simple de donner des nouvelles directement à l'oral plutôt qu'à l'écrit. Nos utilisateurs s'appellent avec cette méthode environ une fois par semaine tandis qu'ils utilisent le chat pratiquement tous les jours.~\\

Pour nos utilisateurs, la photo est le moyen idéal pour faire découvrir à leur entourage leur environnement actuel. Ainsi, pour partager les photos qu'ils ont réalisés, ils ont besoin d'un outil simple de partage de photos. Pour ce faire, nos utilisateurs utilisent deux outils : Facebook et les mails. Le premier leur permet de partager des photos avec leur famille mais aussi avec tous leurs amis tandis que le deuxième leur permet d'envoyer des photos seulement aux personnes voulues. Lorsque les parents demandent des photos ou lorsque l'étudiant à des jolies photos à faire partager, il sélectionne les photos qu'il veut partager et les envoie avec un des moyens cité précédemment. C'est une tâche courte, qui est effectuée plus ou moins souvent en fonction de la quantité de photos à partager. On peut dire que c'est plutôt l'étudiant qui effectue cette tâche et les parents qui regardent les photos mais il peut arriver que les parents envoient eux aussi des photos. Ils procèdent alors de la même manière.~\\

La dernière tâche qu'il est possible de réaliser sur un ordinateur pour communiquer est la création d'un blog ou d'un site web. Pour l'étudiant, c'est un moyen de partager son voyage, ses activités et aussi ses photos. Cette tâche est une communication plus unilatérale mais elle permet de donner des nouvelles aux parents qui peuvent par la suite réagir dans le blog sous forme de commentaires. Faire un blog demande  un peu plus de connaissances et de temps que les tâches précédentes mais il existe aujourd'hui des outils très simples pour réaliser des carnets de voyage ou des articles. Ainsi, si le temps lui permet, l'étudiant crée son blog et met des articles et des photos au fur et à mesure. Les parents peuvent consulter son blog à tout moment pour voir si de nouveaux articles ont été postés. ~\\

Les quatre possibilités de communication décrites ci-dessus sont importantes pour les parents et les enfants car elles permettent de garder un contact même quand des centaines de kilomètres les séparent. L'ordinateur est un outil qui apporte du confort pour appeler et communiquer mais cependant, il a quand même plusieurs défauts qui peuvent empêcher une communication. En effet, il est parfois difficile à transporter et, pour les étudiants et les parents, il est compliqué d'avoir toujours son ordinateur avec soi. De plus, toutes les tâches décrites précédemment demandent une connexion internet Wi-Fi ou Ethernet et, dans certains endroits, cette connexion est impossible à recevoir. L'ordinateur n'est donc pas l'outil idéal pour communiquer à tout moment pour nous qui sommes aujourd'hui devenu des accros de la communication. Heureusement, il existe un outil pour pallier à ce problème : le smartphone.

\subsection{Sur un smartphone}
Les smartphones d'aujourd'hui ne sont ni plus ni moins que des petits ordinateurs que l'on peut transporter partout avec nous. Ils possèdent aussi l'avantage d'avoir accès, grâce au réseau mobile, à la 3G ou à la 4G et par conséquent accès à internet pratiquement tout le temps. Même si le smartphone n'offre pas le même confort que l'ordinateur, les tâches décrites ci-dessus peuvent aussi bien être effectuées avec un smartphone. Dans cette partie, nous détaillerons donc seulement les tâches que les utilisateurs effectuent avec leur smartphone et qu'ils ne peuvent effectuer avec leur ordinateur. ~\\

Les applications de chats sont les plus présentes et les plus utilisées sur les téléphones portables. Ainsi, on retrouve Facebook Messenger, Skype et les mails que l'étudiant et ses parents peuvent utiliser à la fois sur l'ordinateur et sur leur smartphone. Mais sur leur téléphone ils utilisent aussi l'application par défaut pour envoyer des SMS ou encore les applications What's App et Viber. En effet, ces deux dernières permettent de communiquer gratuitement à l'étranger. Le déroulement de cette tâche est encore plus simple sur téléphone : les utilisateurs ouvrent l'application, sélectionnent le contact à qui ils veulent envoyer un message textuel, rédigent le message et l'envoient. L'utilisateur qui reçoit le message est alors averti par un signal sonore, une vibration ou une notification textuelle et il peut y répondre immédiatement ou plus tard. Grâce à ces applications les parents et l'étudiant peuvent communiquer quand ils le souhaitent. Ils se partagent leur emploi du temps, ce qu'ils sont en train de faire, leur état de santé et plein d'autres choses. ~\\
 
L'utilité première d'un portable est de passer des coups de fil. En comparaison avec l'appel par ordinateur, il n'est pas nécessaire pour les deux utilisateurs de se mettre d'accord sur la date et l'heure du coup de fil. En effet, si un des utilisateurs souhaite contacter l'autre il lui suffit de passer son appel et son interlocuteur, s'il est disponible, accepte l'appel. Pour passer des appels sur un smartphone il existe évidemment une application native dans le téléphone,  mais on retrouve également Skype et enfin, l'application Viber qui permet, en plus d'échanger des messages, de passer des appels gratuitement à l'étranger.~\\

Les appels vidéo sont également possibles car aujourd'hui tous les smartphones sont équipés d'une caméra frontale. Grâce à Skype parents et enfants peuvent se voir pratiquement tout le temps. Il existe aussi FaceTime, une application dédiée aux utilisateurs d'iOS et d'iPhone qui permet tout comme Skype de passer des appels vidéo. Les utilisateurs utilisent cette fonctionnalité quand ils n'ont pas accès à leurs ordinateurs pour s'appeler et se voir en même temps.~\\

Enfin, les smartphones d'aujourd'hui sont également équipés d'appareils photos. Cette fonctionnalité permet aux utilisateurs de s'envoyer des photos plus rapidement encore qu'avec un ordinateur. Ces clichés permettent aux parents et aux enfants de partager des moments qu'ils vivent séparément. Une fois que la photo est prise, il est possible pour l'étudiant de la publier sur Facebook comme nous l'avons détaillé dans la section précédente, et les parents découvrent la photo en se connectant sur leur compte. Mais il existe aussi une application qui permet d'échanger des photos en temps réel : Snapchat. L'étudiant prend une photo et l'envoie à ses parents qui reçoivent une notification pour aller regarder la photo. Encore une fois, l'exemple prit ici peut très bien s'appliquer dans l'autre sens : les parents envoient une photo et l'enfant la reçoit. Ce partage de moment est beaucoup utilisé par des personnes loin l'une de l'autre. 

Ainsi, nos utilisateurs utilisent de nombreux logiciels et de nombreuses applications pour garder le contact. Le fait de pouvoir écrire facilement à la personne qui est loin, lui parler de vive voix de temps en temps, pouvoir le voir comme s'il était à côté de soi ou encore avoir une idée de son environnement grâce à des photos est une véritable chance pour une famille qui a besoin de garder le contact. Cependant, ces utilisateurs ont tout de même ressenti certains manques et remarqué quelques failles dans ce système que nous allons décrire par la suite.


\section{Trouver un titre}

\end{document}