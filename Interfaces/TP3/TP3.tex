\documentclass[11pt]{article}
\usepackage[T1]{fontenc}
\usepackage[utf8]{inputenc}
\usepackage{graphicx}
\usepackage{minitoc}
\usepackage[french]{babel}
\usepackage[right=2.5cm, bottom=2.5cm,top=2.5cm, left=2.5cm]{geometry}
\title{\vspace{\fill} Interface et Multimédia \\ ~\textbf{IFT-215} \\~\\ Travail Pratique 3}
\author{Amandine Fouillet - 14 130 638 ~\\ Frank Chassing - 14 153 710 ~\\ Laurent Sénécal-Léonard - 14 143 484}
\date{\today \vspace{\fill}}

\begin{document}
\maketitle
\newpage \thispagestyle{empty}
\null
\newpage
\tableofcontents
\listoffigures
\newpage
\section*{Introduction \markboth{INTRODUCTION}{}}
\addcontentsline{toc}{section}{Introduction}

\section{Représentation globale de l'interface}
Dans cette première partie, nous allons décrire les grandes lignes de l'interface ainsi que les liens entre les différentes fenêtres de l'application. Tous les éléments graphiques de l'interface qui sont décrits dans ce rapport seront représentés visuellement dans l'annexe de ce rapport. Les dessins de l'interface ont été réalisés avec un logiciel et représentent notre application dans sa version OSX. Cependant, l'application pourra également être réalisée sur Windows et les supports mobiles comme iOS et Android. 
\subsection{Identité graphique}
Nous avons choisi de donner le nom "Awaï" à notre application. Ce nom rappelle le mot anglais "Away" qui signifie "Loin" et qui se rapporte à l'idée d'éloignement des familles qui utiliseront l'application. Nous avons changé le "y" par un "ï" pour ne pas utiliser le mot original et laisser un peu de mystère quant à la signification du nom. 

Pour le logo, nous avons choisi une icône d'avion. Simple à représenter, et pas trop encombrant, ce logo rappelle également la notion d'éloignement et de distance entre les utilisateurs.

Pour donner une cohérence entre toutes les fenêtres de l'application nous avons choisit de créer une petite identité graphique. Pour ce faire, nous avons décidé de reprendre des formes elliptiques  sur les différentes pages et d'appliquer un code couleur pastel (Figure \ref{fig:codecouleur}).
\subsection{La fenêtre de connexion}
Lors de la première utilisation, juste après son installation, l'application se lancera sur une fenêtre de connexion (Figure \ref{fig:connexion})pour mettre à l'utilisateur de s'identifier et ainsi d'avoir accès à son espace personnel. Cette fenêtre comporte peu d'éléments mis à par le nom de l'application et son logo en haut au centre, trois boutons situés au milieu de la page : Connexion, Inscription et À Propos et enfin deux boutons iconiques en bas à gauche représentant les paramètres et l'aide. 

L'utilisateur pourra choisir de cliquer sur le bouton "Connexion" s'il est déjà inscrit à l'application, il rentrera alors ses identifiants et accédera à son espace. Au contraire, si c'est la première fois qu'il utilise l'application il sélectionnera "Inscription" et rentrera ses coordonnées pour se créer un compte. Enfin, il peut souhaiter en savoir plus sur l'application et aller explorer ce qui se trouve derrière le bouton "À Propos".  Les rôles des liens vers les paramètres et l'aide seront expliqués plus en détails dans la section \ref{par:aide}.

\subsection{Architecture de la fenêtre principale}
Sur la fenêtre principale  (Figure \ref{fig:principale}), comme sur la fenêtre de connexion, on retrouve en haut au centre le nom de l'application et son logo ainsi que les icônes de paramètres et d'aide en bas à gauche. Au centre on retrouve deux rangées de trois ellipses colorées qui contiennent les icônes des fonctionnalités de l'application. Sur la première rangée on retrouve la communication (section \ref{par:com}), les photos (section \ref{par:photos}) et le calendrier (section \ref{par:cal}) tandis que sur la seconde on retrouve le suivi des dépenses (section \ref{par:depenses}), la carte du monde (section \ref{par:carte}) et les contacts (section \ref{par:contact}). En dessous de chaque ellipse représentant une fonctionnalité, on trouve une courte description textuelle de la fonctionnalité pour aider la compréhension des utilisateurs qui ne sauraient pas ce que l'icône signifierait.

\subsection{Lien entre les fenêtres}
Dès que l'on sélectionne une fonctionnalité en cliquant sur une des ellipses, on quitte la fenêtre principale. Le décor diffère en fonction de la fonctionnalité dans laquelle on se trouve mais afin que l'utilisateur ne soit pas perdu, nous avons conservé une base graphique (Figure \ref{fig:base}) que l'on retrouvera dans chaque fonctionnalité. En effet, sur chaque fenêtre on retrouvera en haut à gauche une icône de maison pour le retour à l'accueil, les six ellipses représentant les fonctionnalités de notre application alignées les unes en dessous des autres sur le côté gauche de la fenêtre et, enfin, les icônes d'aide et de paramètres côte à côte en bas à gauche.
En fonction de la fonctionnalité dans laquelle l'utilisateur se trouve, une des ellipses aura disparu et seule l'icône de la fonctionnalité restera visible. De plus, le titre, l'ellipse et l'icône de la fonctionnalité seront rappelés en haut de la fenêtre.

Sur certaines fenêtres, celles des fonctionnalités photos, calendrier, dépenses et carte on retrouve une icône de partage en haut à gauche. Cette icône permet à l'utilisateur de sélectionner les amis à qui il veut partager cette fonctionnalité. En effet, il peut par exemple vouloir partager des choses différentes à ses parents qu'à ses amis. Cet outil lui permet de contrôler l'accès de ses amis à ses données personnelles. 

Pour naviguer d'une fonctionnalité à une autre il suffit simplement de cliquer sur son icône dans la barre permanente de gauche. On change alors de décors pour atterrir dans la nouvelle fonctionnalité. Sur un ordinateur, pour les utilisateurs qui veulent garder plusieurs vues de l'application ouverte il est également possible d'ouvrir d'autres fenêtres en sélectionnant l'option nouvelle fenêtre dans le menu de l'application.

\section{Description des fonctionnalités}
\subsection{La communication}\label{par:com}
	Tout d’abord, cette fonctionnalité regroupe trois moyens de communication différents, soit les appels  audio, les appels audio et vidéo ou par messages instantanés. Ceux-ci seront libres aux choix des utilisateurs. Pour les communications par appels, le fonctionnement de notre application sera semblable à celui de l’application Skype. En d’autres termes, les utilisateurs devront se mettre d’accord sur la date et l’heure à laquelle ils veulent se contacter. Il sera possible d’envoyer des messages instantanés à d’autres contacts lors des appels  et il sera possible d’inviter davantage de contact lors des communications. Puis, lorsqu’un utilisateur reçoit une invitation pour amorcer une conversation audio ou audio et vidéo, une alerte surviendra sur l’interface du récepteur demandant si oui ou non celui-ci désire commencer cette discussion. 
	
	Ensuite, l’interface de cette fonctionnalité se décrit par une liste contenant les conversations récentes où apparait le nom du contact et sa photo sont affichés pour chacune des discussions. De plus, il est affiché pour chacun des contacts le dernier message instantané reçu ou envoyé. Pour tous les contacts présents dans la liste, il y a à la droite de chacun d’eux un bouton ayant un téléphone de dessiné dans celui-ci et un deuxième bouton où une camera vidéo y est dessiné. Ainsi, pour commencer une conversation audio, l’utilisateur doit cliquer sur le bouton «téléphone» et pour amorcer une conversation audio et vidéo l’utilisateur doit cliquer sur le bouton «caméra vidéo». Finalement, pour écrire un message instantané l’émetteur devra double-cliquer sur le contact désiré.
\subsection{Le partage de photos}\label{par:photos}
Cette fonctionnalité permettra à l’utilisateur de partager des photos qu’il désire montrer à son entourage. Lorsque ce dernier prend des photos, il lui sera possible de les publier dans cette fonctionnalité. Il pourra classer ses photos selon l’évènement en question ou selon la date où la photo a été prise. Afin que son entourage puisse voir les photos enregistrés dans le répertoire de cette fonctionnalité, l’utilisateur devra sélectionner les photos et cliquer sur partager avec le ou les contacts désirés.

Au niveau de l’interface,  chaque répertoire de photos se présente par une bulle munie d’une photo de l’évènement à l’intérieur et du nom de l’évènement situé à l’extérieur de la bulle. Pour ajouter un répertoire de photo, un bouton (+) se trouver sur la barre de menu qui permettra de faire l’action. L’utilisateur pourra donner un nom qu’il a choisi ou simplement ne pas donner de nom ce qui affichera par défaut la date de prise de photos. Finalement, un bouton de partage se trouve sur la barre des menus permettant à l’utilisateur de partager ses photos sélectionnées avec ses contacts.
\subsection{L'emploi du temps}\label{par:cal}
Tout d’abord,  cette fonctionnalité permet à l’utilisateur d’afficher la gestion de son temps à son entourage. Les informations ajoutés en entré seront enregistré dans un calendrier où il sera possible de choisir précisément l’heure de la journée en question. Par soucis de confidentialité, l’utilisateur pourra choisir a son aise les informations qu’il désire partager et avec le contact en particulier.

Concernant l’interface, par défaut  celui-ci se présente avec un calendrier affichant le mois en question et ses journées numérotées. Lorsqu’il y a  un ajout d’un évènement dans le calendrier, la journée en question paraitra dans un en encadrer rouge. Pour ajouter des éléments dans le calendrier, l’utilisateur devra faire un double-clique sur  la journée en question. Il sera possible à l’utilisateur de changer la disposition du calendrier selon un format année où il sera afficher les douze mois et il sera possible d’afficher le calendrier sous une forme hebdomadaire. Finalement, un bouton pour partager se trouve dans le menu de la fonctionnalité.
\subsection{Le suivi des dépenses}\label{par:depenses}
\subsection{La carte du monde}\label{par:carte}
Cette fonctionnalité n'a pas été évoquée directement dans le TP2 mais nous avons voulu la rajouter de manière à ce que les personnes éloignées puissent continuer à se suivre à distance. En effet, cette carte du monde permettra à l'enfant d'indiquer avec des punaises les endroits où il s'est rendu, les lieux qu'il a visités. De cette manière, ses parents, ses grands-parents pourront suivre son chemin et ses aventures à distance. 

Au niveau de l'interface, cette fonctionnalité se présente simplement comme une carte du monde que l'on peut parcourir, zoomer et dézoomer. Grâce à un appui plus long sur une zone de la carte, l'utilisateur peut rajouter une "punaise" pour signaler à ses amis qu'il s'est rendu dans cet endroit. Il choisit la couleur de la punaise en fonction de la raison de sa visite (une punaise rouge pour le lieu d'habitation, une punaise bleue pour les lieux scolaires et une punaise verte pour les visites).
\subsection{Les contacts}\label{par:contact}
\subsection{Les paramètres et l'aide en ligne}\label{par:aide}

\newpage
\section*{Conclusion \markboth{CONCLUSION}{}}
\addcontentsline{toc}{section}{Conclusion}

\newpage
\section*{Annexes \markboth{ANNEXES}{}}
\addcontentsline{toc}{section}{Annexes}
\begin{figure}[hbtp]
        \centering \includegraphics[scale=0.6]{Modelisation/Couleurs/pastels.jpg}
        \caption{Code couleur}
	\label{fig:codecouleur}
\end{figure}
\begin{figure}[hbtp]
    \begin{minipage}[b]{0.4\linewidth}
        \centering \includegraphics[scale=0.43]{Modelisation/connexion.png}
        \caption{Fenêtre de connexion}
	\label{fig:connexion}
    \end{minipage}\hfill
    \begin{minipage}[b]{0.48\linewidth}
        \centering \includegraphics[scale=0.43]{Modelisation/awai.png}
       \caption{Fenêtre principale}
\label{fig:principale}
    \end{minipage}
\end{figure}
\begin{figure}[hbtp]
    \begin{minipage}[b]{0.4\linewidth}
        \centering \includegraphics[scale=0.43]{Modelisation/base.png}
        \caption{Base de l'interface}
\label{fig:base}
    \end{minipage}\hfill
    \begin{minipage}[b]{0.48\linewidth}
        \centering \includegraphics[scale=0.43]{Modelisation/communication.png}
        \caption{Gestion des communications}
        \label{fig:communication}
    \end{minipage}
\end{figure}
\begin{figure}[hbtp]
    \begin{minipage}[b]{0.4\linewidth}
        \centering \includegraphics[scale=0.43]{Modelisation/photos.png}
        \caption{Gestion des photos}
                \label{fig:photos}
\label{fig:base}
    \end{minipage}\hfill
    \begin{minipage}[b]{0.48\linewidth}
        \centering \includegraphics[scale=0.43]{Modelisation/calendrier.png}
        \caption{Gestion du calendrier}
        \label{fig:calendrier1}
    \end{minipage}
\end{figure}
\begin{figure}[hbtp]
    \begin{minipage}[b]{0.4\linewidth}
        \centering \includegraphics[scale=0.43]{Modelisation/calendrier2.png}
        \caption{Gestion du calendrier}
                \label{fig:calendrier2}
\label{fig:base}
    \end{minipage}\hfill
    \begin{minipage}[b]{0.48\linewidth}
        \centering \includegraphics[scale=0.43]{Modelisation/depenses.png}
        \caption{Gestion des dépenses}
         \label{fig:depenses}
    \end{minipage}
\end{figure}
\begin{figure}[hbtp]
    \begin{minipage}[b]{0.4\linewidth}
        \centering \includegraphics[scale=0.43]{Modelisation/carte.png}
        \caption{Gestion de la carte}
                \label{fig:carte}
\label{fig:base}
    \end{minipage}\hfill
    \begin{minipage}[b]{0.48\linewidth}
        \centering \includegraphics[scale=0.43]{Modelisation/contact.png}
        \caption{Gestion des contacts}
         \label{fig:contact}
    \end{minipage}
\end{figure}


\end{document}