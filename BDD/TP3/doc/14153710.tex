\documentclass[11pt]{article}
\usepackage[T1]{fontenc}
\usepackage[utf8]{inputenc}
\usepackage{graphicx}
\usepackage[french]{babel}
\usepackage[right=2cm, bottom=2cm,top=1cm, left=2cm]{geometry}
\title{Exploitation de bases de données relationnelles et OO ~\\~\textbf{IFT-287} \\~\\ Rapport Individuel TP3}
\author{Frank Chassing - 14 153 710}
\date{\today}

\begin{document}
\maketitle
Afin de développer le programme du TP3, nous avons repris la structure de l’exemple de la bibliothèque. Nous avons donc créé des classes principales ayant des fonctions permettant d’accéder aux différentes tables et de récupérer les informations nécessaires. Nous avons programmés également d’autres classes de gestion contenant les différentes fonctions appelées lors de l’exécution du Main. Nous avons aussi repris la notion de tuples,  ce qui nous a permis de garder la même structuration de l’exemple et accéder plus facilement et précisément aux différentes informations voulues. Nous possédons également une classe Main permettant soit de lire un fichier contenant des commandes, soit de lire des commandes entrées directement par l’utilisateur.~\\

En ce qui concerne la distribution des tâches, nous avons séparé le travail en deux pour ce qui est de la programmation des différentes classes. Amandine s’est occupée des classes concernant tout ce qui était en rapport avec Match, Arbitre et les résultats. Elle a également effectué les fonctions demandées (de 7 à 13) concernant ces éléments. De plus, elle a programmé la classe Séquence permettant de gérer les fonctions d’incrémentations d’ID dans les tables. Pour ma part, j’ai travaillé sur tout ce qui était en rapport avec Equipe, Joueur et Terrain. J’ai donc réalisé les fonctions directement liées à ces éléments (de 1 à 6). Chacun de notre côté nous avons effectué des tests pour chacune de nos fonctions, puis nous nous sommes retrouvés pour programmer ensemble la méthode Main et pour ensuite faire les différentes phases de tests finales de notre application. Celle-ci est complète et pourrait éventuellement comprendre d’autres fonctionnalités plus poussées. En effet, l’architecture est d’ores et déjà écrite et la totalité des fonctions d’accès de base est implémentée.~\\

Ce TP a été fort intéressant dans la mesure où c’est la première fois que je couple du SQL avec le langage Java. Je pense également qu’il est important de savoir maîtriser davantage le langage Java et SQL car ce sont des langages fortement utilisés de nos jours. C’est pourquoi ce TP a été enrichissant et cette façon d’utiliser une base de donnée via le langage Java me sera sûrement d’une forte utilité à l’avenir.




\end{document}